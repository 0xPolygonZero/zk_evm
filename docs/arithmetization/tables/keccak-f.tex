\subsection{Keccak-f}
\label{keccak-f}

This table computes the Keccak-f[1600] permutation.

\subsubsection{Keccak-f Permutation}
To explain how this table is structured, we first need to detail how the permutation is computed. \href{https://keccak.team/keccak_specs_summary.html}{This page} gives a pseudo-code for the permutation. Our implementation differs slightly -- but remains equivalent -- for optimization and constraint degree reasons. 

Let:
\begin{itemize}
    \item $S$ be the sponge width ($S=25$ in our case)
    \item $\texttt{NUM\_ROUNDS}$ be the number of Keccak rounds ($\texttt{NUM\_ROUNDS} = 24$)
    \item $RC$ a vector of round constants of size  $\texttt{NUM\_ROUNDS}$
    \item $I$ be the input of the permutation, comprised of $S$ 64-bit elements
\end{itemize}    

The first step is to reshape $I$ into a $5 \times 5$ matrix. We initialize the state $A$ of the sponge with $I$: $$A[x, y] := I[x, y] \text{ }  \forall x, y \in \{0..4\}$$

We store $A$ in the table, and subdivide each 64-bit element into two 32-bit limbs.
Then, for each round $i$, we proceed as follows:
\begin{enumerate}
    \item First, we define $C[x] := \texttt{xor}_{i=0}^4 A[x, i]$. We store $C$ as bits in the table. This is because we need to apply a rotation on its elements' bits and carry out \texttt{ xor } operations in the next step.
    \item Then, we store a second vector $C'$ in bits, such that: $$C'[x, z] = C[x, z] \texttt{ xor } C[x-1, z] \texttt{ xor } C[x+1, z-1]$$. 
    \item We then need to store the updated value of $A$: $$A'[x, y] = A[x, y] \texttt{ xor } C[x, y] \texttt{ xor } C'[x, y]$$ Note that this is equivalent to the equation in the official Keccak-f description: $$A'[x, y] = A[x, y] \texttt{ xor } C[x-1, z] \texttt{ xor } C[x+1, z-1]$$.
    \item The previous three points correspond to the $\theta$ step in Keccak-f. We can now move on to the $\rho$ and $\pi$ steps. These steps are written as: $$B[y, 2\times x + 3 \times y] := \texttt{rot}(A'[x, y], r[x, y])$$ where $\texttt{rot(a, s)}$ is the bitwise cyclic shift operation, and $r$ is the matrix of rotation offsets. We do not need to store $B$: $B$'s bits are only a permutation of $A'$'s bits. 
    \item The $\chi$ step updates the state once again, and we store the new values: $$A''[x, y] := B[x, y] \texttt{ xor } (\texttt{not }B[x+1, y] \texttt{ and } B[x+2, y])$$ Because of the way we carry out constraints (as explained below), we do not need to store the individual bits for $A''$: we only need the 32-bit limbs.
    \item The final step, $\iota$, consists in updating the first element of the state as follows: $$A'''[0, 0] = A''[0, 0] \texttt{ xor } RC[i]$$ where $$A'''[x, y] = A''[x, y] \forall (x, y) \neq (0, 0)$$ Since only the first element is updated, we only need to store $A'''[0, 0]$ of this updated state. The remaining elements are fetched from $A''$. However, because of the bitwise $\texttt{xor}$ operation, we do need columns for the bits of $A''[0, 0]$.  
\end{enumerate}

Note that all permutation elements are 64-bit long. But they are stored as 32-bit limbs so that we do not overflow the field. 

It is also important to note that all bitwise logic operations ($\texttt{ xor }$, $\texttt{ not }$ and $\texttt{ and}$) are checked in this table. This is why we need to store the bits of most elements. The logic table can only carry out eight 32-bit logic operations per row. Thus, leveraging it here would drastically increase the number of logic rows, and incur too much overhead in proving time.



\subsubsection{Columns}
Using the notations from the previous section, we can now list the columns in the table:
\begin{enumerate}
    \item $\texttt{NUM\_ROUND}S = 24$ columns $c_i$ to determine which round is currently being computed. $c_i = 1$ when we are in the $i$-th round, and 0 otherwise. These columns' purpose is to ensure that the correct round constants are used at each round.
    \item $1$ column $t$ which stores the timestamp at which the Keccak operation was called in the cpu. This column enables us to ensure that inputs and outputs are consistent between the cpu, keccak-sponge and keccak-f tables.
    \item $5 \times 5 \times 2 = 50 $columns to store the elements of $A$. As a reminder, each 64-bit element is divided into two 32-bit limbs, and $A$ comprises $S = 25$ elements.
    \item $5 \times 64 = 320$ columns to store the bits of the vector $C$.
    \item $5 \times 64 = 320$ columns to store the bits of the vector $C'$.
    \item $5 \times 5 \times 64 = 1600$ columns to store the bits of $A'$.
    \item $5 \times 5 \times 2 = 50$ columns to store the 32-bit limbs of $A''$.
    \item $64$ columns to store the bits of $A''[0, 0]$.
    \item $2$ columns to store the two limbs of $A'''[0, 0]$.
\end{enumerate}

In total, this table comprises 2,431 columns.

\subsubsection{Constraints}
Some constraints checking that the elements are computed correctly are not straightforward. Let us detail them here.

First, it is important to highlight the fact that a $\texttt{xor}$ between two elements is of degree 2. Indeed, for $x \texttt{ xor } y$, the constraint is $x + y - 2 \times x \times y$, which is of degree 2. This implies that a $\texttt{xor}$ between 3 elements is of degree 3, which is the maximal constraint degree for our STARKs.

We can check that $C'[x, z] = C[x, z] \texttt{ xor } C[x - 1, z] \texttt{ xor } C[x + 1, z - 1]$. However, we cannot directly check that $C[x] = \texttt{xor}_{i=0}^4 A[x, i]$, as it would be a degree 5 constraint. Instead, we use $C'$ for this constraint. We see that:
$$\texttt{xor}_{i=0}^4 A'[x, i, z] = C'[x, z]$$
This implies that the difference $d = \sum_{i=0}^4 A'[x, i, z] - C'[x, z]$ is either 0, 2 or 4. We can therefore enforce the following degree 3 constraint instead:
$$d \times (d - 2) \times (d - 4) = 0$$

Additionally, we have to check that $A'$ is well constructed. We know that $A'$ should be such that $A'[x, y, z] = A[x, y, z] \texttt{ xor } C[x, z] \texttt{ xor } C'[x, z]$. Since we do not have the bits of $A$ elements but the bits of $A'$ elements, we check the equivalent degree 3 constraint:
$$A[x, y, z] = A'[x, y, z] \texttt{ xor } C[x, z] \texttt { xor } C'[x, z]$$

Finally, the constraints for the remaining elements, $A''$ and $A'''$ are straightforward: $A''$ is a three-element bitwise $\texttt{xor}$ where all bits involved are already storedn and $A'''[0, 0]$ is the output of a simple bitwise $\texttt{xor}$ with a round constant.